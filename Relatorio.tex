%%%%%%%%%%%%%%%%%%%%%%%%%%%%%%%%%%%%%%%%%%%%%%%%%%%%%%%%%%%%%%%%%%%%%%%%%%%%%%%

\documentclass[12pt, a4paper, oneside]{article}

\input{/home/giatro/.config/user/giatro_packages.tex}

\input{/home/giatro/.config/user/giatro_macros.tex}

\title{Relatório - EP2 - Modelagem e Simulação}
\date{\today}
\author{
Lucas Paiolla Forastiere \\ 11221911 
\and
Davi de Menezes Pereira \\ 11221988
}

%%%%%%%%%%%%%%%%%%%%%%%%%%%%%%%%%%%%%%%%%%%%%%%%%%%%%%%%%%%%%%%%%%%%%%%%%%%%%%%
%%%%%%%%%%%%%%%%%%%%%%%%%%%%%%%%%%%%%%%%%%%%%%%%%%%%%%%%%%%%%%%%%%%%%%%%%%%%%%%

\begin{document}

\maketitle
\newpage

\section{Sobre as decisões iniciais}

A primeira coisa que fizermos foi ler o enunciado e buscamos entender as equações
e, em especial, a última. Depois, criamos um cronograma com a ordem das tarefas:

\begin{enumerate}
\item Fazer a função de Euler para aproximar a equação diferencial;
\item Fazer a função para criar o vetor da parte 1 (com o número de casos por
tempo);
\item Utilizar essa função para fazer os gráficos referentes à parte 1;
\item Criar uma função que gera uma ilha randomicamente para a parte 2;
\item Utilizar a função da primeira parte para fazer a segunda (que criar uma matriz
em que cada linha é um vetor de número de casos por tempo);
\item Criar uma função que pega a matriz da função anterior e calcula um vetor 
de normas dos vetores formados pelas colunas da matriz;
\item Fazer a parte que plota os gráficos da parte 2;
\item Fazer as animações desses gráficos;
\item Decidir se faríamos ou não a parte extra;
\item Fazer o relatório.
\end{enumerate}

Depois de fazer esse cronograma, nós decidimos como dividiríamos as tarefas.

\subsection{Sobre a divisão de tarefas}

Dividimos as tarefas da seguinte maneira:

\begin{enumerate}
\item Lucas;
\item Lucas;
\item Davi;
\item Lucas;
\item Lucas;
\item Lucas
\item Davi;
\item Davi;
\item Lucas (relatório).
\end{enumerate}

\section{Sobre as decisões de implementação}

De começo, já tínhamos decididos a maioria das funções nas decisões iniciais.

Após isso, as principais decisões de implementação foram:

\begin{enumerate}
\item Os parâmetros utilizados pela função de resoluçao da equação diferencial. 
Nós passamos os antigos valores de $N$ e $T$, o valor do $\Delta T$ e um 
vetor de parâmetros que são utilizados na equação diferencial, eles são:
\begin{enumerate}
    \item \textsc{param[0]} representa o $\alpha$ da equação;
    \item \textsc{param[1]} representa o $\lambda$ da equação;
    \item \textsc{param[2]} representa o $t_0$ da equação;
    \item \textsc{param[3]} representa o $\mu$ da equação.
\end{enumerate}

\item Fazer a função da parte 1 receber os parâmetros citados acima (utilizados
pela função de Euler). Assim, poderíamos utilizar ela cinco vezes apenas mudando
esses parâmetros na parte 2;

\item A decisão de começar os primeiros valores da equação diferencial como 
$N = 1$ e $t = 1$, onde $N$ é o número de casos e $t$ é a unidade de tempo em dias.

\item A decisão de usar um $dt$ igual a $0.1$. Fizemos alguns testes e avaliamos 
que essa seria uma boa decisão.

\item A decisão de usar o tempo de final da simulação como $t_f = 200$. Ou seja,
duzentos dias após o começo da pandemia. Parece condiser com o que observamos 
na realidade (como, por exemplo, o Brasil, que já está com aproximadamente
cem dias desde o caso zero e ainda não vimos um pico claro);

\item Uma decisão importante e demorada foi a de escolher os intervalos em que
geraríamos os parâmetros aleatoriamente para as ilhas da parte 2. A função 
\textsc{CriaIlha} gera um vetor de quatro coordenadas (de acordo com a decisão 1.)
com os parâmetros nos seguintes intervalos:
\begin{eqnarray}
0.15 &<~ \alpha &< 0.45 \\ 0.001 &<~ \lambda &< 0.003 \\ 20 &<~ t_0 &< 80 \\ 10 &<~ \mu &< 100
\end{eqnarray}

Fizemos uma série de testes, mantendo três parâmetros constantes e variando o
outro para enfim chegar na conclusão de que esses intervalos eram bons e 
condisentes com a realidade (dentro dos limites esperados da simulação).

\item Outra decisão foi a forma como fizemos a matriz da parte dois, onde em 
cada linha temos um vetor igual ao da parte1. Essa decisão afeta a forma como 
plotamos esses gráficos depois e também como criamos o vetor das normas.

\item Por fim, tomamos a decisão de salvar as animações em um arquivo a parte, ao
invés de mostrar no notebook, pois estavamos enfrentando alguns problemas com 
a quantidade de memória disponível nos notebooks.
\end{enumerate}



%%%%%%%%%%%%%%%%%%%%%%%%%%%%%%%%%%%%%%%%%%%%%%%%%%%%%%%%%%%%%%%%%%%%%%%%%%%%%%%

\end{document}
